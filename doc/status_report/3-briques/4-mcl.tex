	\subsection{Génération des lignes centrales}

Une fois les fichiers de paires de points définis, ils sont utilisés chacun à leur tour avec la surface extraite pour générer les lignes centrales avec \execname{meshing\_centerlines}.


	\subsubsection{\ioT}

\iolist
{fichier \stl + fichier \pointpair ou fichier lisible par gmsh (.geo ou maillage 1D .msh).}
{fichier \vtk.}


	\subsubsection{\argsT}

\args
{config-file}{}{string}{nom du fichier de configuration à utiliser}
{input.surface.filename}{}{string}{fichier \stl}
{input.pointpair.filename}{}{string}{fichier \pointpair}
{output.directory}{}{string}{répertoire de sortie du fichier \vtk à produire}
{delaunay-tessellation.force-rebuild}{0}{bool}{si le fichier existe déjà: 1 pour forcer la tessellation de Delaunay (et écraser le fichier), 0 pour la désactiver}
{delaunay-tessellation.output.directory}{}{string}{répertoire de sortie de la tessellation de Delaunay}
\stoparg


	\subsubsection{\etatg}
	

