	\subsection{Génération d'image 3D}

À partir des lignes centrales fusionnées, on génère une nouvelle image 3D (isotrope) avec \execname{meshing\_imagefromcenterlines}.

	\subsubsection{\ioT}

\iolist
{fichier \vtk.}
{fichier \mha.}


	\subsubsection{\argsT}

\args
{config-file}{}{string}{nom du fichier de configuration à utiliser}
{input.centerlines.filename}{}{string}{fichier \vtk contenant les lignes centrales}
{output.directory}{}{string}{répertoire pour le fichier \mha à produire}
{dim.x}{0}{int}{|}
{dim.y}{0}{int}{\} dimensions de l'image à produire (en voxels)}
{dim.z}{0}{int}{|}
{dim.spacing}{0.0}{double}{espacement entre les points (taille des voxels)}
{radius-array-name}{MaximumInscribedSphereRadius}{string}{nom de la liste des rayons de sphères inscrites}
\stoparg


	\subsubsection{\etatg}

La génération d'image fonctionne bien. Il est important de noter que la taille des pixels demandée doit être inférieure au rayon minimum.
