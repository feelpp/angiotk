	\subsection{Filtrage d'image 3D: \rorpo}
	\label{brique:rorpo}
	
\rorpo est utilisé via l'exécutable \execname{RORPO\_multiscale\_usage}, qui permet de cumuler les exécutions de \rorpo pour plusieurs échelles consécutives.

	\subsubsection{Prétraitements}

Quelques traitements et vérifications sont effectuées avant de lancer \rorpo:

\begin{itemize}
\item Conversion au format \mha si nécessaire (\texttt{vmtkimagereader/vmtkimagewriter}).
\item Création d'une version isotrope si nécessaire (KitWare/itk: \texttt{ResampleVolumesToBeIsotropic}).
\item Conversion au format \nii (\texttt{vmtkimagereader/vmtkimagewriter}).
\end{itemize}

	\subsubsection{\ioT}

\iolist
{image 3D isotrope au format \mha ou \nii (le choix entre les deux n'a aucune incidence).}
{image 3D isotrope au format \mha ou \nii (à choisir via le paramètre \texttt{-{}-extension}).}



	\subsubsection{\argsT}
	\label{brique:rorpo:p}

\args
{input (-i)}{}{string}{nom du fichier image à traiter.}
{output (-o)}{}{string}{nom du fichier à produire.}
{extension (-e)}{}{int}{format de fichier en sortie, 0 pour \mha, 1 pour \nii.}
{smin (-s)}{}{int}{taille minimale de chemin (en pixels).}
{factor (-f)}{}{float}{raison de la suite géométrique des échelles: $s_{n+1} = f \cdot s_n$.}
{nb\_scale (-n)}{}{int}{nombre d'échelles à utiliser (taille de la suite).}
{core (-f)}{}{int}{nombre de processus (threads) à utiliser (maximum 7).}
\stoparg



%Note: $s_{min}$ est donnée en pixels. On peut calculer ce paramètre en fonction de la taille minimale des structures à mettre en évidence (voir l'annexe \ref{sec:lminRORPO})


	\subsubsection{\etatg}
\rorpo fonctionne sur l'ensemble des images disponibles, et aucun problème n'est à signaler.

	\subsubsection{Post-traitements}

L'image 3D obtenue ne subit qu'un seul post-traitement avant l'étape suivante: une reconversion au format \mha (\texttt{vmtkimagereader/vmtkimagewriter}).


