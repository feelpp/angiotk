	\subsection{Traitement et maillage de volume}

Dans cette derni�re �tape, \execname{meshing\_volumefromstlandcenterlines} permet de mailler le volume des vaisseaux sanguins. Il est possible d'extruder plusieurs couches au niveau de la paroi, afin notamment de prendre en compte son �paisseur lors des simulations d'interactions fluide-structure. On va aussi pouvoir marquer les diff�rentes entr�es et sorties du maillage afin de pouvoir faire des simulations, en g�n�rant des fichiers \desc contenant des nom de marqueurs et les positions des points correspondants.

	\subsubsection{\ioT}
	
\iolist
{fichier \stl, fichier \vtk et �ventuellement fichier \desc}
{fichier \msh}

\subsubsection{Fichier \desc}

Si aucun fichier \desc n'est fourni, il sera g�n�r�. Si le fichier est fourni, il sera utilis� pour g�n�rer le maillage final.
	
	\subsubsection{\argsT}

\args
{config-file}{}{string}{Nom du fichier de configuration � utiliser}
{input.surface.filename}{}{string}{Fichier \stl (la surface � traiter)}
{input.centerlines.filename}{}{string}{Fichier \vtk (les lignes centrales)}
{output.directory}{}{string}{R�pertoire de sortie du fichier \stl � produire}
\stoparg

		\subsubsection{Fichier de configuration}

\configfile
{force-rebuild}{1}{bool}{0 pour �viter le calcul si le fichier cible existe d�j�, 1 pour le forcer (et �craser le fichier cible)}
{output.save-binary}{1}{bool}{D�finit le type de fichier \msh � produire: 1 pour du binaire, 0 pour de l'ASCII}
{input.desc.filename}{}{string}{Nom du fichier de description � utiliser \$cfgdir/data/FOA42.desc}
{nb-points-in-circle}{15}{int}{Nombre de points d�sir� pour une section circulaire lors du maillage du volume}
{extrude-wall}{0}{bool}{1 pour extruder une ou plusieurs couches de la paroi, 0 sinon}
{extrude-wall.nb-elt-layer}{2}{int}{Nombre de couches � extruder}
{extrude-wall.h-layer}{0.2}{double}{Epaisseur de la paroi � extruder (en pourcentage du rayon)}
\stopfile

	\subsubsection{\etatg}
	
Si la surface contient plus d'une composante connexe, l'algorithme entre dans une boucle infinie. Une solution est de s�parer les composantes connexes au pr�alable, par exemple avec \paraview, puis de les traiter s�par�ment.


