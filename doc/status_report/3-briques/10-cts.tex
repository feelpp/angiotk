	\subsection{Segmentation par Component Tree}

Afin de contourner le problème des lignes centrales, l'une des pistes envisagées consiste à améliorer la segmentation qui a lieu après le filtrage afin d'extraire directement une surface satisfaisante. Aussi, une méthode de segmentation par component tree a été ajoutée et peut déjà être testée. Celle-ci fait appel à des marqueurs, placés sur les régions d'intérêt de l'image à segmenter. Le principe de la méthode est décrit dans \cite{Passat2011}. La génération de marqueurs est assurée par \rorpo et la segmentation en elle-même par \cts.

	\subsubsection{Génération de marqueurs}

Les marqueurs à utiliser sont donnés par \rorpo: il s'agit précisément des structures tubulaires qu'il met en valeur. Concrètement ces marqueurs sont donc obtenus sous la forme d'une image de mêmes dimensions que l'image à segmenter.

	\subsubsection{\ioT}
	
\iolist
{2 fichiers \nii (image et marqueurs)}
{fichier \nii}

	
	\subsubsection{\argsT}

Les paramètres doivent être donnés dans un ordre précis lors de l'appel de la ligne de commande:

\begin{lstlisting}
$ component_tree_segmentation <input path> <label path> <output path> <alpha> <threshold>
\end{lstlisting}

\args
{<input path}{}{string}{Image à segmenter}
{<label path}{}{string}{Marqueurs à utiliser (= image filtrée par \rorpo)}
{<output path}{}{string}{Répertoire de sortie}
{<alpha}{}{double}{Paramètre $\alpha$ (nombre réel) compris entre 0 et 1}
{<threshold}{}{int}{Seuil (nombre entier) compris entre 0 et 255}
\stoparg

	\subsubsection{\etatg}
	

