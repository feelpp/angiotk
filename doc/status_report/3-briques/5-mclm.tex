	\subsection{Fusion des lignes centrales}

Les lignes centrales sont finalement fusionnées avec \execname{meshing\_centerlinesmanager}.

	\subsubsection{\ioT}

\iolist
{fichier \stl, fichiers \vtk.}
{fichier \vtk.}


	\subsubsection{\argsT}

\args
{config-file}{}{string}{nom du fichier de configuration à utiliser.}
{input.surface.filename}{}{string}{fichier \stl (optionnel, sert à estimer le rayon des lignes centrales).}
{input.centerlines.filename}{}{string}{fichier \vtk d'une ligne centrale (un paramètre à ajouter pour chaque fichier à fusionner).}
{output.directory}{}{string}{répertoire pour le fichier \vtk à produire.}
\stoparg


	\subsubsection{Fichier de configuration}

\configfile
{remove-branch-ids}{}{vector of int}{pour retirer certaines branches}
%{use-window-interactor}{0}{bool}{}
%{window-width}{1024}{}{}
%{window-height}{768}{}{}
{[threshold-radius]}{}{}{}
{min}{-1}{double}{rayon minimum}
{max}{-1}{double}{rayon maximum}
{[avoid-tubular-colision]}{}{}{}
{apply}{0}{bool}{1 pour activer le mécanisme, 0 sinon}
{distance-min}{0.4}{double}{distance minimale}
{radius-min}{0.4}{double}{rayon minimum}
{[smooth-resample]}{}{}{}
{apply}{1}{bool}{1 pour activer le mécanisme, 0 sinon}
{geo-points-spacing}{4.0}{double}{espace entre les points}
{mesh-size}{1.0}{double}{}
\stopfile

	\subsubsection{\etatg}
	
Des collisions sont possibles selon le rayon et la proximité des vaisseaux sanguins. Ajuster les paramètres \argfont{avoid-tubular-colision.*} permet d'éviter ce phénomène.

